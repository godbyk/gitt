% afterhours8.tex - Afer Hours Week 8
\chapter{After Hours Week 8}
\section{``Fishing for beginners''}
\subsection{Hooking your scripts up}
So we come to the last After Hours section of the book and this time we are going to take a look at hook scripts.
Hook scripts are tiny pieces of code that you are allowed to have executed at certain points during Git's processes which can affect the outcome of those same processes.
Imagine for example that whenever you committed to a repository you wanted an email to be sent to the project manager.
That is a job for a hook script.

The are a fairly large number of hook points available.
They are called hooks because it allows you to hook something else into the Git process.
Below is a list of a few of the available hook points along with a very brief description of when they are triggered and how they can affect the process.
For a full list, you should check out the manual for githooks.

\begin{itemize}
\item\textbf{applypatch-msg} - Invoked by \texttt{git am}, it is provided with the name of the proposed commit message for the patch and allows for sanitisation.
Exiting with a non-zero status will cause \texttt{git am} to abort the patch.
Will run the \textbf{commit-msg} hook if enabled.
\item\textbf{pre-applypatch} - Invoked by \texttt{git am}, it runs immediately after the patch is applied, but before a commit is made.
In this way, checks can be made to the working tree before committing.
Exiting with non-zero status aborts the commit.
Will run the \textbf{pre-commit} hook if enabled.
\item\textbf{post-applypatch} - Invoked by \texttt{git am}, it runs after the patch is applied and the commit is made.
It does not affect the flow of \texttt{git am} and as such is useful for actions such as notifications.
\item\textbf{pre-commit} - This hook is invoked by \texttt{git commit} and can be bypassed with the \texttt{--no-verify} option.
Exiting with non-zero status aborts the commit.
\item\textbf{prepare-commit-msg} - This hook is invoked by \texttt{git commit} and is run immediately after the commit message has been prepared, but before the message editor has been loaded.
It takes three parameters, the file containing the log message, the source of the commit message and an SHA hash.
Exiting with non-zero status aborts the commit.
It should not be used to replace \textbf{pre-commit}.
\item\textbf{commit-msg} - This hook is invoked by \texttt{git commit} and can be bypassed with the \texttt{--no-verify} option.
The hook is used to normalise messages in place before commit.
Exiting with non-zero status aborts the commit.
\item\textbf{post-commit} - Invoked by \texttt{git commit}, it runs immediately after a commit is made.
It can not affect the flow of \texttt{git commit} and as such is useful for actions such as notifications.
\item\textbf{pre-rebase} - Invoked by \texttt{git rebase}, it is useful for preventing a branch from being rebased.
\item\textbf{post-checkout} - Invoked by \texttt{git checkout} after it has updated the working tree.
The hook is given the ref of the previous HEAD, the new HEAD and a flag indicating if it was a branch or file checkout.
It is also invoked after \texttt{git clone}, but can be overidden by the use of \texttt{--no-checkout}.
The script is most useful for performing sanity and validity checks on the working tree post checkout.
It does not affect the process of \texttt{git clone} or \texttt{git checkout}
\item\textbf{post-merge} - Invoked by \texttt{git merge} when run as part of a \texttt{git pull}.
It does not affect the outcome of the merge, but does not get executed if the merge fails due to conflicts.
\end{itemize}

So we have a large arsenal of hooks that we can employ to assist us in administrating our repository.
We are going to set up a fairly simple example here of ensuring that we do not delete one of our core files.
To begin with, let us add a \texttt{corefile} to the repository in a new branch and play a little with a new hook script.

\begin{code}
john@satsuki:~/coderepo$ git checkout -b hooktest
Switched to a new branch 'hooktest'
john@satsuki:~/coderepo$ echo "Very Important File" > corefile
john@satsuki:~/coderepo$ git add corefile
john@satsuki:~/coderepo$ git commit -a -m 'Added Core File'
[hooktest 96fe8b9] Added Core File
 1 files changed, 1 insertions(+), 0 deletions(-)
 create mode 100644 corefile
john@satsuki:~/coderepo$
\end{code}

Now let us create a hook script called \texttt{.git/hooks/pre-commit}, with the following code.

\begin{code}
#!/bin/bash

if [ -e "corefile" ] 
then
	echo "File OK"
	exit 0
else
	echo "File Deleted Not Committing"
	exit 2
fi
\end{code}

Now let us see what happens if we make a commit that does not affect the \texttt{corefile}.

\begin{code}
john@satsuki:~/coderepo$ echo "New Development" >> cont_dev 
john@satsuki:~/coderepo$ git commit -a -m 'Added a commit'
File OK
[hooktest 73656f2] Added a commit
 1 files changed, 1 insertions(+), 0 deletions(-)
john@satsuki:~/coderepo$
\end{code}

We see the text \texttt{File OK} in the output, which indicates that the commit is allowed to go ahead.
What happens if we try to remove the core file and commit the result?

\begin{code}
john@satsuki:~/coderepo$ git rm corefile
rm 'corefile'
john@satsuki:~/coderepo$ git commit -a -m 'Removed a corefile'
File Deleted Not Committing
john@satsuki:~/coderepo$ 
\end{code}

We have been prevented from committing because we removed the all important \texttt{corefile}.
To be able to commit again, we must return the file.

\begin{code}
john@satsuki:~/coderepo$ git checkout HEAD -- corefile
john@satsuki:~/coderepo$ ls
another_file  cont_dev  corefile  tester
john@satsuki:~/coderepo$ echo "New development" >> cont_dev 
john@satsuki:~/coderepo$ git commit -a -m 'Some more development'
File OK
[hooktest 6786dd5] Some more development
 1 files changed, 1 insertions(+), 0 deletions(-)
john@satsuki:~/coderepo$
\end{code}

So you see hooks can be really simple, or they can do highly complex things.
Imagine a hook that would compile the code in the working tree prior to the commit and only allow it to continue if the code compiled cleanly.
That could be pretty useful in ensuring that there is always good clean buildable code in the repository.
